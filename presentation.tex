\documentclass[10pt,xcolor=dvipsnames]{beamer}

\usepackage[utf8]{inputenc}
\usepackage[T1]{fontenc}
\usepackage[french]{babel}

\usepackage{graphicx}
\usepackage{amsmath, amssymb, amsthm, dsfont, mathrsfs, mathtools}
\usepackage{tikz-cd}
\usepackage{geometry}
\usepackage{hyperref}
\usepackage{fancyhdr}
\usepackage{caption}
\usepackage{ulem}

\usetheme[progressbar=frametitle]{metropolis}
\usepackage{appendixnumberbeamer}

\usepackage{booktabs}
\usepackage[scale=2]{ccicons}

\usepackage{pgfplots}
\usepgfplotslibrary{dateplot}

\usepackage{xspace}
\newcommand{\themename}{\textbf{\textsc{metropolis}}\xspace}
\usepackage[export]{adjustbox}

\usepackage{pdfpages}
\setbeamercolor{section title}{fg=Maroon,bg=Maroon}
\setbeamercolor*{structure}{bg=Maroon!20,fg=Maroon}
\setbeamercolor*{palette primary}{use=structure,fg=white,bg=structure.fg}
\setbeamercolor{progress bar}{fg=gray, bg=gray}


\usepackage{minted}
\usepackage{ebproof}

\usepackage{tikz}
\usetikzlibrary{positioning, fit}
\usetikzlibrary{calc}
\usetikzlibrary{shapes.geometric, arrows.meta}

\newcommand{\init}{\iota}
\newcommand{\Init}{\mathcal {I}}
\newcommand{\ancestor}{\mathrel{\sqsubseteq}}

\definecolor{red}{rgb}{1,0,0}
\definecolor{green}{rgb}{0,1,0}
\definecolor{darkgreen}{rgb}{0,0.5,0}
\definecolor{blue}{rgb}{0,0,1}
\definecolor{violet}{rgb}{0.4,0,0.8}

\newcommand{\red}[1]{\textcolor{red}{#1}}
\newcommand{\green}[1]{\textcolor{green}{#1}}
\newcommand{\darkgreen}[1]{\textcolor{darkgreen}{#1}}
\newcommand{\blue}[1]{\textcolor{blue}{#1}}
\newcommand{\violet}[1]{\textcolor{violet}{#1}}


\newcommand{\parts}[1]{\mathcal{P}\left(#1\right)}

\theoremstyle{plain}
\newtheorem{theoreme}{Théorème}[section]
\newtheorem{definitionn}[theoreme]{Définition}

\begin{document}

\begin{frame}[plain]
	\centering
	\vspace{1cm}

	{\huge \textbf{Décidabilité de l’uniformisation MSO sur des arbres finis}}

	\vspace{0.5cm}
	{\Large Thomas Colcombet \& \underline{Yago Iglesias Vázquez}}


	\vspace{1cm}
	\begin{minipage}{0.4\textwidth}
		\centering
		\includegraphics[height=1cm]{images/logo.jpg}

		Université Paris Cité
	\end{minipage}

	\vfill
	{\small \today}
\end{frame}



\begin{frame}{Arbres binaires finis non ordonnés}

	\begin{itemize}
		\item On travaille avec des \textbf{arbres binaires finis non ordonnés} :
		      \begin{itemize}
			      \item Chaque nœud a deux fils.
			      \item On \textbf{ne distingue pas} entre fils gauche et droit.
		      \end{itemize}

		      \vspace{0.8em}

		\item Question centrale : \\
		      \textit{Si les deux sous-arbres sont identiques, comment choisir en choisir un?}
	\end{itemize}

	\vspace{1em}

	\centering
	\begin{tikzpicture}[
			level distance=1.2cm,
			sibling distance=1.4cm,
			every node/.style={circle,draw,minimum size=6mm,inner sep=1pt},
			text height=1.5ex, text depth=.25ex
		]
		\node {a}
		child {node {b}}
		child {node {b}};
	\end{tikzpicture}
\end{frame}


\begin{frame}{Logique MSO — Aperçu rapide}
	\begin{itemize}
		\item Permet de décrire des propriétés sur les arbres, \textbf{sans distinguer gauche/droite}.
		\item Variables :
		      \begin{itemize}
			      \item \textbf{1er ordre} : nœuds ($x, y, \dots$)
			      \item \textbf{2nd ordre} : ensembles de nœuds ($X, Y, \dots$)
		      \end{itemize}
		\item Quantification possible sur les deux types : $\forall x$, $\exists X$, etc.
		\item Exemple : $\varphi(x) = \text{``$x$ est une feuille''}$
		      \[
			      \scalebox{0.7}{
				      \begin{tikzpicture}[sibling distance=1.2cm, level distance=1.1cm,
						      every node/.style = {circle, draw, minimum size=0.6cm}, baseline={(current bounding box.center)}]

					      \node {$a$}
					      child {node {$c$}}
					      child {node[draw=blue, thick] {$b$}};
				      \end{tikzpicture}
			      } \models\ \varphi\left(\tikz[baseline=-0.5ex]\node[draw=blue, circle, inner sep=1pt]{\strut b};\right)
		      \]
	\end{itemize}
\end{frame}

\begin{frame}{Définissabilité}
	\begin{definitionn}
		Un ensemble de nœuds $X$ d’un arbre $t$ est \textbf{définissable} s'il existe une formule $\psi(x)$ telle que :
		\[
			X = \{ x \mid t \models \psi(x) \}.
		\]
	\end{definitionn}
\end{frame}


\begin{frame}{Exemples de définissabilité}
	\centering
	\textbf{$\varphi_1(X) = \text{``$X$ singleton avec une feuille''}$}
	\vspace{0.5em}

	\begin{tabular}{cc}
		\begin{tikzpicture}[level distance=1.2cm, sibling distance=1.2cm,
				every node/.style = {circle, draw, minimum size=0.7cm}]
			\node[fill=blue!20] {$a$};
		\end{tikzpicture}
		                         &
		\begin{tikzpicture}[level distance=1.2cm, sibling distance=1.3cm,
				every node/.style = {circle, draw, minimum size=0.7cm}]
			\node {$a$}
			child {node {$b$}}
			child {node {$b$}};
		\end{tikzpicture}
		\\[1ex]
		\darkgreen{Définissable} & \red{Pas définissable}
	\end{tabular}

	\vspace{2em}
	\pause

	\textbf{$\varphi_2(X) = \text{``$X$ contient au moins une feuille''}$}

	\vspace{0.5em}

	\begin{tikzpicture}[level distance=1.2cm, sibling distance=1.3cm,
			every node/.style = {circle, draw, minimum size=0.7cm}]
		\node {$a$}
		child {node[fill=blue!20] {$b$}}
		child {node[fill=blue!20] {$b$}};
	\end{tikzpicture}\\
	\vspace{0.3em}
	\darkgreen{Définissable}
\end{frame}


\begin{frame}{Uniformisation — Définition}

	\begin{itemize}
		\item $\varphi(X)$ est \textbf{uniformisable} s’il existe une formule $\psi(x)$ telle que pour tout arbre $t$ :\\
		      Si $t \models \varphi(X)$, alors $X_\psi := \{x \mid t \models \psi(x)\}$ vérifie $t \models \varphi(X_\psi)$.
		      \vspace{1em}
	\end{itemize}

\end{frame}

\begin{frame}{Uniformisation — Exemples}
	\begin{itemize}
		\item $\varphi_1(X) = \text{``$X$ singleton avec une feuille''}$

		      \begin{itemize}
			      \item Existence d’un arbre sans solution définissable :\\
			            \vspace{1em}
			            \begin{center}
				            \scalebox{0.7}{
					            \begin{tikzpicture}[level distance=1.2cm, sibling distance=1.2cm,
							            every node/.style = {circle, draw, minimum size=0.7cm}]
						            \node {$a$}
						            child {node {$b$}}
						            child {node {$b$}};
					            \end{tikzpicture}
				            }
			            \end{center}
			      \item \red{Pas uniformisable}
		      \end{itemize}

		      \vspace{1em}

		      \pause

		\item $\varphi_2(X) = \text{``$X$ contient au moins une feuille''}$

		      \begin{itemize}
			      \item $\psi(x) = \text{``$x$ est une feuille''}$.
			      \item \darkgreen{Uniformisable}
		      \end{itemize}
	\end{itemize}
\end{frame}


\begin{frame}{Uniformisation — Problème}
	\begin{itemize}
		\item \textbf{Problème :} Étant donné une formule $\varphi(X)$, décider si elle est uniformisable.
	\end{itemize}
\end{frame}

\begin{frame}{Algorithme d’uniformisation — Idée générale}

	\begin{enumerate}
		\item Entrée : une formule $\varphi(X)$.
		\item Construire l’automate uniformiseur $U$ associé à $\varphi(X)$.
		\item Tester si $U$ est universel.
		\item Retourner \texttt{Oui} si $U$ est universel, \texttt{Non} sinon.
	\end{enumerate}

	\begin{itemize}
		\item Résultat clé : \\
		      \vspace{0.3em}
		      \hspace{1em} $U$ universel \quad $\Longleftrightarrow$ \quad $\varphi(X)$ est uniformisable.
	\end{itemize}

\end{frame}



\begin{frame}{Automates finis bottom-up déterministes (DBUA)}

	\begin{itemize}
		\item Un DBUA est un 5-uplet $ A = (\Sigma, Q, \init, \delta, F) $
		\item On impose $ \delta_a(\blue p, \red q) = \delta_a(\red q,\blue p)$
	\end{itemize}

	\vspace{1em}

	\centering
	\scalebox{0.85}{
		\begin{tikzpicture}[
				level distance=1.2cm,
				sibling distance=1.4cm,
				every node/.style={circle,draw,minimum size=6mm,inner sep=1pt},
				state/.style={blue},
				text height=1.5ex, text depth=.25ex,
				arrowstyle/.style={draw=none, font=\normalsize}
			]

			% Étape 0 : Arbre original
			\node at (-4.5,0) {a}
			child {node {b}}
			child {node {d}
					child {node {c}}
					child {node {b}}};

			\pause
			\node[arrowstyle] at (-2.5, 0) {$\displaystyle\Rightarrow_{\init}$};

			% Étape 1 : feuilles étiquetées
			\node at (-0.5,0) {a}
			child {node[state] {1}}
			child {node {d}
					child {node[state] {0}}
					child {node[state] {1}}};

			\pause
			\node[arrowstyle] at (1.7, 0) {$\displaystyle\Rightarrow_{\delta_d(0,1)}$};

			% Étape 2
			\node at (4,0) {a}
			child {node[state] {1}}
			child {node[state] {1}};

			\pause
			\node[arrowstyle] at (5.8, 0) {$\displaystyle\Rightarrow_{\delta_a(1,1)}$};

			% État final
			\node[state] at (7,0) {2};

		\end{tikzpicture}
	}

\end{frame}

\begin{frame}{MSO et automates : équivalence}\
	\begin{theoreme}
		Pour tout langage $L$ d'arbres, il existe un DBUA ordre-insensible $A$ qui reconnaît $L$ ssi il existe $\varphi(X)$ une formule MSO telle que,
		$L = \{ t \mid t \models \varphi(X) \}$.
	\end{theoreme}
\end{frame}


\begin{frame}{Arbres annotés}
	\begin{itemize}
		\item Pour $\varphi(X)$, il existe un DBUA $A$ sur $\Sigma \times \{0,1\}$ tel que :
		      \[
			      t \models \varphi(X) \text{ ssi } (t,X) \text{ est accepté par } A.
		      \]
		\item Arbre annoté $(t,X)$ : chaque nœud de $t$ est étiqueté par $(\sigma,b)$ avec $b=1$ si le nœud appartient à $X$, sinon $b=0$.\\
		      \vspace{1em}
		      \begin{center}
			      \scalebox{1}{
				      \begin{tikzpicture}[level distance=1.2cm, sibling distance=1.2cm,
						      every node/.style={circle, draw, minimum size=0.7cm, inner sep=1pt}]
					      \node (root) {a, \textcolor{blue}{1}}
					      child {node {b, \textcolor{blue}{0}}}
					      child {node {c, \textcolor{blue}{1}}};
				      \end{tikzpicture}
			      }
		      \end{center}
		      \vspace{1em}
	\end{itemize}
\end{frame}


\begin{frame}{Powerset — Construction de l'automate}

	\centering
	\begin{tabular}{ll}
		\toprule
		\textbf{Automate} & \textbf{Définition}                                                          \\
		\midrule
		$A$               & $A : \text{Tree}_{\Sigma \times \{0,1\}} \to Q$                              \\
		                  & $A(t,X) \in F$ \text{ ssi } $t \models \varphi(X)$                           \\
		\\
		$PA$ (powerset)   & $PA : \text{Tree}_{\Sigma} \to \parts Q$                                     \\
		                  & $PA(t) = \{ A(t,X) \mid X \subseteq \mathrm{Nodes}(t) \}$                    \\
		\\
		                  & $PA(t) \cap F \neq \emptyset$ \text{ ssi } $t \models \exists X. \varphi(X)$ \\
		\bottomrule
	\end{tabular}

	\vspace{1.5em}
	\begin{itemize}
		\item $PA$ simule $A$ sur toutes les assignations possibles de $X$.
	\end{itemize}

\end{frame}

\begin{frame}{Automate uniformiseur — Idée générale}

	\begin{itemize}
		\item Cherche à construire une solution de $\varphi(X)$ \textbf{et} une solution \textbf{définissable}.
		\item L'automate accepte les arbres vérifiant :
		      \[
			      ''\exists X.\, \varphi(X) \implies \exists X.\, \varphi(X) \text{ avec } X \text{ définissable}''
		      \]
		\item $U$ universel ssi pour tout arbre $t$, si $t \models \exists X.\, \varphi(X)$, alors il existe une solution \textbf{définissable}.
	\end{itemize}

\end{frame}

\begin{frame}{Automate uniformiseur — Construction}

	\begin{tabular}{ll}
		\toprule
		$A : \text{Tree}_{\Sigma \times \{0,1\}} \to Q$                     & $A(t,X) \in F$ ssi $t \models \varphi(X)$.                                                             \\
		\\
		$PA : \text{Tree}_{\Sigma} \to \mathcal{P}(Q)$                      & $PA(t) = \{ A(t,X) \mid X \subseteq \text{Nodes}(t) \}$                                                \\
		                                                                    & $PA(t) \cap F \neq \emptyset \text{ ssi } t \models \exists X. \varphi(X)$                             \\
		\midrule
		\\
		$U : \text{Tree}_{\Sigma} \to \mathcal{P}(Q) \times \mathcal{P}(Q)$ & $U(t) = (P,P')$                                                                                        \\
		                                                                    & $P \cap F \neq \emptyset \implies P' \cap F \neq \emptyset$    ssi   $t \models \exists X. \varphi(X)$ \\
		                                                                    & implique qu'il existe une solution définissable                                                        \\
		\bottomrule
	\end{tabular}

	\vspace{1em}

	\begin{itemize}
		\item $P$ correspond à l'exécution de $PA$.
		\item $P'$ correspond à une exécution contrainte de $PA$ qui garantit l'existence d'une solution définissable.
	\end{itemize}

\end{frame}


\begin{frame}{Conclusion et Perspectives}
	\begin{itemize}
		\item Introduction pédagogique aux automates sur arbres et à la logique MSO.
		      \vspace{0.7em}
		\item \textbf{Le résultat principal :} l’uniformisation de la logique MSO sur les arbres binaires finis non ordonnés est décidable.
		      \vspace{0.7em}
		\item Perspectives futures :
		      \begin{itemize}
			      \item Étendre ces techniques aux arbres \textit{unranked}.
		      \end{itemize}
	\end{itemize}
\end{frame}


\begin{frame}[shrink=15]{Références}
	\nocite{*}
	\bibliographystyle{plain}
	\bibliography{tree-uniformisation}
\end{frame}



\end{document}

